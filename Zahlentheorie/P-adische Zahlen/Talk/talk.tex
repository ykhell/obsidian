\documentclass[a4paper]{article}
\usepackage{amsfonts}
\usepackage{amssymb}
\usepackage{amsmath}
\usepackage{graphicx}
\usepackage{amsthm}
\usepackage{geometry}
 \geometry{
 a4paper,
 total={170mm,257mm},
 left=30mm,
 right=30mm,
 top=10mm,
 bottom=10mm,
 }
\graphicspath{ {./graphics/} }
\title{\vspace{-2cm}Strassman's theorem and the Logarithm Function}
\author{Yousef Khell}

\theoremstyle{plain}
\newtheorem{thm}{Theorem}[section]
\newtheorem{lemm}[thm]{Lemma}
\newtheorem{prop}[thm]{Proposition}
\newtheorem{theo}[thm]{Theorem}
\newtheorem{Cor}[thm]{Corollary}
\newtheorem{thm-defi}[thm]{Theorem-Definition}

\theoremstyle{definition}
\newtheorem{defi}[thm]{Definition}
\newtheorem{rem}[thm]{Remark}
\newtheorem{ex}[thm]{Example}
\newtheorem{Conc}[thm]{Conclusion}
\newtheorem{nota}[thm]{Notation}
\newtheorem{exer}[thm]{Exercise}
\newtheorem{soln}[thm]{Solution}
\usepackage{fancyhdr}

\begin{document}
\section{Revision}
\begin{prop}{\cite[Prop. 5.1.4]{Gou}}
  Let $b_{ij} \in \Bbb Q_{p}$ and suppose $\forall i: \lim_{j \to \infty}b_{ij} = 0$ and $lim_{i \to \infty}b_{ij} = 0$ uniformly in $j$, then both series
  $\sum_{i=0}^{\infty}\left(\sum_{j=0}^{\infty} b_{ij} \right) \text{ and } \sum_{j=0}^{\infty}\left(\sum_{i=0}^{\infty} b_{ij} \right)$
  converge and have equal sum.
\end{prop}
\section{Strassman's Theorem}
 % Let $f \in \Bbb Q_{p}[[X]]$ a formal power series over $\Bbb Q_{p}$, then for $x, y \in \Bbb Z_{p}$ we have
 % $$f(x) - f(y) = (x-y)\sum_{n=1}^{\infty}\sum_{j=0}^{n-1}a_{n}x^{j}y^{n-1-j}$$
 % $$f(x) - f(y) = \sum_{n=0}^{\infty}a_{n}x^{n} - \sum_{n=0}^{\infty}a_{n}y^{n} = \sum_{n=1}^{\infty}a_{n}(x^{n}-y^{n})$$
\begin{rem} % lemma 2.1
  Let $(R, +, \cdot)$ be a ring, $x, y \in R$, then we have $x^{n}-y^{n} = (x-y)\sum_{j=0}^{n-1}x^{j}y^{n-1-j}, \forall n \in \Bbb N_{0}$
  % TODO Ask if this should be in terms of rings or what
\end{rem}
\begin{proof} % proof 2.1
    , we use induction on $n$,
    Base case: $n = 2$, it's easy to see that
    $$(x-y)\sum_{j=0}^{n-1}x^{j}y^{n-1-j} = (x-y)(x+y) = x^{2} - y^{2}$$
    Induction hypothesis: we assume for an arbitrary $n \geq 2: x^{n}-y^{n} = (x-y)\sum_{j=0}^{n-1}x^{j}y^{n-1-j}$,
    Induction step: consider
    $$(x-y)\sum_{j=0}^{n}x^{j}y^{n-j} = (x-y)(y^{n} + y^{n-1}x + \cdots + x^{n-1}y + x^{n})$$
    $$ = (x-y)(y(y^{n-1} + y^{n-2}x + \cdots + yx^{n-2} + x^{n-1}) + x^{n}) = y\underbrace{(x-y)\sum_{j=0}^{n-1}x^{j}y^{n-1-j}}_{= x^{n}-y^{n}} + x^{n}(x-y)$$
    $$ = y(x^{n} - y^{n}) + x^{n}(x-y) = yx^{n} - y^{n+1} + x^{n+1} - yx^{n} = x^{n+1} - y^{n+1}.$$
\end{proof}
\begin{lemm}
  Let $f(X) \in \Bbb Q_{p}[[X]]$ be a non-zero power series which converges $\forall x \in \Bbb Z_{p}$, then $\exists N \in \Bbb N_{0}$ such that $|a_{N}| = \max_{n \in \Bbb N_{0}}|a_{n}| \text{ and } |a_{n}| < |a_{N}|\ \forall n > N$
\end{lemm}
\begin{proof}
  Since $f(X)$ converges $\forall x \in \Bbb Z_{p}$, then we have
  $$\forall x \in \Bbb Z_{p} : \lim_{n \to \infty}|a_{n}x^{n}| = 0 = \lim_{n \to \infty}|a_{n}|\cdot|x^{n}| \implies \lim_{n \to \infty}|a_{n}| = 0$$
\end{proof}
\begin{thm}[Strassman]
Let $f(X) \in \Bbb Q_{p}[[X]]$ and suppose we have $\lim_{n \to \infty}a_{n} = 0$, so that $f(x)$ converges $\forall x \in \Bbb Z_{p}$. Define $N \in \Bbb N_{0}$ like in Lemma 2.2
then the function $f$ has at most $N$ zeros.
\end{thm}
\begin{proof}
  induction on $N$.
  \begin{itemize}
    \item Base case: if $N = 0$, then $|a_{0}| > |a_{n}|, \forall n \geq 1$, we want to show that there are no zeros: $f(x) \neq 0 \forall x \in \Bbb Z_{p}$, if we had $f(x) = 0$, then
          $$0 = f(x) = a_{0} + a_{1}x + a_{2}x^{2} + \cdots$$
          $$ \implies |a_{0}| = |a_{1}x + a_{2}x^{2} + \cdots| \leq \max_{n \geq 1}|a_{n}x^{n}| \leq \max_{n \geq 1}|a_{n}|$$
          But this contradicts the assumption that $|a_{0}| > |a_{n}|, \forall n \geq 1$, so there are no zeros in this case.
    \item Induction step:
          Suppose $N$ was defined like before, and $\exists \alpha \in \Bbb Z_{p} : f(\alpha) = 0$, then we have for any $x \in \Bbb Z_{p}$
          $$f(x) = f(x) - f(\alpha) = \sum_{n=0}^{\infty}a_{n}x^{n} - \sum_{n=0}^{\infty}a_{n}\alpha^{n} = \sum_{n=0}^{\infty}a_{n}(x^{n} - \alpha^{n}) \overset{2.1}= (x-\alpha)\sum_{n=0}^{\infty}\sum_{j = 0}^{n-1} a_{n}x^{j}\alpha^{n-1-j}$$
          $$ = (x-\alpha) \sum_{n=0}^{\infty} \sum_{j=0}^{\infty} c_{nj}, \quad c_{nj}:= \begin{cases}a_{n}x^{j}\alpha^{n-1-j} & j < n, \\ 0 & j \geq n.\end{cases}$$
          We can use prop 1.1 to change the order of the summation but first we have to show the conditions of the proposition:
          \begin{enumerate}
            \item $\forall n \in \Bbb N_{0}, \lim_{j \to \infty} c_{nj} = 0$:
                  Clear, since we have $c_{nj} = 0, \forall j \geq n$.
            \item $\lim_{n \to \infty}c_{nj} = 0$ uniformly in $j$:
                  This is also easy to see, because we have $|a_{n}x^{j}\alpha^{n-1-j}| \leq |a_{n}| \to 0$ unrelated to $j$.
          \end{enumerate}
          So we can switch the sums and then we have
          $$(x-\alpha) \sum_{n=0}^{\infty} \sum_{j=0}^{\infty} c_{nj} = (x-\alpha) \sum_{j=0}^{\infty} \sum_{n=0}^{\infty} c_{nj}$$
          since $\forall j \geq n:c_{nj} = 0$, we need to only consider when $n > j$ so its equal to
          $$= (x-\alpha) \sum_{j=0}^{\infty} \sum_{n=j+1}^{\infty} a_{n}x^{j}\alpha^{n-1-j} = (x-\alpha) \sum_{j=0}^{\infty} x^{j} \underbrace{\sum_{n=0}^{\infty} a_{n+j+1}\alpha^{n}}_{=:b_{j}}$$
          $$= (x-\alpha)g(x),\quad g(x):=\sum_{j=0}^{\infty}b_{j}x^{j}$$
  \end{itemize}
\end{proof}
\begin{Cor}
  Let $f(X) = \sum a_{n}x^{n}$ be a non-zero power series which converges on $\Bbb Z_{p}$, and let $\alpha_{1}, ..., \alpha_{m} \in \Bbb Z_{p}$ be the roots of $f(X)$ in $\Bbb Z_{p}$, then there exists another power series $g(X)$ which also converges on $\Bbb Z_{p}$ but has no zeros in $\Bbb Z_{p}$, for which
  $$f(X) = \left(\prod_{i=1}^{m}(X-\alpha_{i})\right)g(X)$$
  \begin{proof}

  \end{proof}
\end{Cor}

\begin{Cor}
  Let $f(X) = \sum a_{n}x^{n}$ be a non-zero power series which converges on $p^{m}\Bbb Z_{p}$, for some $m \in \Bbb Z$. Then $f(X)$ has a finite number of roots in $p^{m} \Bbb Z_{p}$.
  \begin{proof}

  \end{proof}
\end{Cor}
\begin{Cor}
  Let $f(X) = \sum a_{n}x^{n}$ and $g(X) = \sum b_{n}X^{n}$ be two p-adic power series which converge in a disc $p^{m}\Bbb Z_{p}$. If there exist infinitely many numbers $\alpha \in p^{m}Z_{p}$ such that $f(\alpha) = g(\alpha)$, then $a_{n} = b_{n}, \forall n \geq 0$

  \begin{proof}

  \end{proof}
\end{Cor}
\begin{Cor}
  Let $f(X) = \sum a_{n}x^{n}$ be a p-adic power series which converges in some disc $p^{m}\Bbb Z_{p}$. If the function $p^{m}\Bbb Z_{p} \to \Bbb Q_{p}, x \mapsto f(x)$ is periodic, that is, $\exists \pi \in p^{m}\Bbb Z_{p} : f(x + \pi) = f(x), \forall \in p^{m}\Bbb Z_{p}$ then $f(X)$ is constant.

  \begin{proof}

  \end{proof}
\end{Cor}
\begin{Cor}
  Let $f(X) = \sum a_{n}x^{n}$ be a p-adic power series which is entire, that is, $f(x)$ converges $\forall x \in \Bbb Q_{p}$. Then $f(X)$ has at most countably many zeros. Furthermore, if the set of zeros is not finite then the zeros form a sequence $\alpha_{n}$ with $|\alpha_{n}| \to \infty$.
  \begin{proof}

  \end{proof}
\end{Cor}
\section{The $p$-adic Logarithm Function}
\section{Roots of Unity}
\begin{thebibliography}{widest-label} % References are listed here

	\bibitem[Gou]{Gou}%You can add more refernces by \bibitem command
	Fernando Q. Gouv\^{e}a:
	\emph{p-adic Numbers}.


\end{thebibliography}
\end{document}


