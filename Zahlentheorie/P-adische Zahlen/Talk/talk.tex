\documentclass[a4paper]{article}
\usepackage{amsfonts}
\usepackage{amssymb}
\usepackage{amsmath}
\usepackage{graphicx}
\usepackage{amsthm}
\usepackage{geometry}
 \geometry{
 a4paper,
 total={170mm,257mm},
 left=30mm,
 right=30mm,
 top=10mm,
 bottom=10mm,
 }
\graphicspath{ {./graphics/} }
\title{\vspace{-2cm}Übungen Analysis 1 - Blatt 3}
\author{Yousef Khell}

\theoremstyle{plain}
\newtheorem{thm}{Theorem}[section]
\newtheorem{lemm}[thm]{Lemma}
\newtheorem{prop}[thm]{Proposition}
\newtheorem{theo}[thm]{Theorem}
\newtheorem{Cor}[thm]{Corollary}
\newtheorem{thm-defi}[thm]{Theorem-Definition}

\theoremstyle{definition}
\newtheorem{defi}[thm]{Definition}
\newtheorem{rem}[thm]{Remark}
\newtheorem{ex}[thm]{Example}
\newtheorem{Conc}[thm]{Conclusion}
\newtheorem{nota}[thm]{Notation}
\newtheorem{exer}[thm]{Exercise}
\newtheorem{soln}[thm]{Solution}
\usepackage{fancyhdr}

\begin{document}
\section{Revision}
\section{Strassman's Theorem}
For the entirety of section §2 let
$$f: \Bbb Z_{p} \to \Bbb Q_{p}, f(X) = \sum_{n=0}^{\infty}a_{n}X^{n} = a_{0} + a_{1}X + a_{2}X^{2} + \cdots$$
be a non-zero (formal?) power series with coefficients in $\Bbb Q_{p}$
\begin{lemm}
  For $x, y \in \Bbb Z_{p}$ we have
  $$f(x) - f(y) = (x-y)\sum_{n=1}^{\infty}\sum_{j=0}^{n-1}a_{n}x^{j}y^{n-1-j}$$
  %$$f(x) - f(y) = (x-y)\sum_{n=1}^{\infty}a_{n}(x^{n-1}+x^{n-2}y + \cdots + xy^{n-2} + y^{n-1})$$
\end{lemm}
\begin{proof}

\end{proof}
\begin{lemm}
If $f(x)$ converges $(\lim_{n\to \infty} a_{n} = 0)$ $\forall x \in \Bbb Z_{p}$, then $\exists N \in \Bbb N_{0}:$
$$|a_{N}| = \max_{n \in \Bbb N_{0}}|a_{n}| \text{ and } |a_{n}| < |a_{N}| \forall n > N$$
\begin{proof}

\end{proof}
\end{lemm}
\begin{thm}[Strassman] %make this non italic
Suppose we have $\lim_{n \to \infty}a_{n} = 0$, so that $f(x)$ converges $\forall x \in \Bbb Z_{p}$. Define $N \in \Bbb N_{0}$ by the following condition:
$$|a_{N}| = \max_{n \in \Bbb N_{0}}|a_{n}| \text{ and } |a_{n}| < |a_{N}| \forall n > N$$
then the function $f$ has at most $N$ zeros.
\begin{proof}
First step is to show that $N$ exists.
\end{proof}
\end{thm}
\begin{Cor}
  Let $f(X) = \sum a_{n}x^{n}$ be a non-zero power series which converges on $\Bbb Z_{p}$, and let $\alpha_{1}, ..., \alpha_{m} \in \Bbb Z_{p}$ be the roots of $f(X)$ in $\Bbb Z_{p}$, then there exists another power series $g(X)$ which also converges on $\Bbb Z_{p}$ but has no zeros in $\Bbb Z_{p}$, for which
  $$f(X) = \left(\prod_{i=1}^{m}(X-\alpha_{i})\right)g(X)$$
  \begin{proof}

  \end{proof}
\end{Cor}

\begin{Cor}
  Let $f(X) = \sum a_{n}x^{n}$ be a non-zero power series which converges on $p^{m}\Bbb Z_{p}$, for some $m \in \Bbb Z$. Then $f(X)$ has a finite number of roots in $p^{m} \Bbb Z_{p}$.
  \begin{proof}

  \end{proof}
\end{Cor}
\begin{Cor}
  Let $f(X) = \sum a_{n}x^{n}$ and $g(X) = \sum b_{n}X^{n}$ be two p-adic power series which converge in a disc $p^{m}\Bbb Z_{p}$. If there exist infinitely many numbers $\alpha \in p^{m}Z_{p}$ such that $f(\alpha) = g(\alpha)$, then $a_{n} = b_{n}, \forall n \geq 0$

  \begin{proof}

  \end{proof}
\end{Cor}
\begin{Cor}
  Let $f(X) = \sum a_{n}x^{n}$ be a p-adic power series which converges in some disc $p^{m}\Bbb Z_{p}$. If the function $p^{m}\Bbb Z_{p} \to \Bbb Q_{p}, x \mapsto f(x)$ is periodic, that is, $\exists \pi \in p^{m}\Bbb Z_{p} : f(x + \pi) = f(x), \forall \in p^{m}\Bbb Z_{p}$ then $f(X)$ is constant.

  \begin{proof}

  \end{proof}
\end{Cor}
\begin{Cor}
  Let $f(X) = \sum a_{n}x^{n}$ be a p-adic power series which is entire, that is, $f(x)$ converges $\forall x \in \Bbb Q_{p}$. Then $f(X)$ has at most countably many zeros. Furthermore, if the set of zeros is not finite then the zeros form a sequence $\alpha_{n}$ with $|\alpha_{n}| \to \infty$.

  \begin{proof}

  \end{proof}
\end{Cor}
\section{The $p$-adic Logarithm Function}
\section{Roots of Unity}
\end{document}
