\documentclass[a4paper]{article}
\usepackage{amsfonts}
\usepackage{amssymb}
\usepackage{amsmath}
\usepackage{graphicx}
\usepackage{amsthm}
\usepackage{geometry}
 \geometry{
 a4paper,
 total={170mm,257mm},
 left=20mm,
 right=20mm,
 top=10mm,
 bottom=10mm,
 }
\graphicspath{ {./graphics/} }
\title{\vspace{-2cm}Strassman's theorem and the Logarithm Function}
\author{Yousef Khell}

\theoremstyle{plain}
\newtheorem{thm}{Theorem}[section]
\newtheorem{lemm}[thm]{Lemma}
\newtheorem{prop}[thm]{Proposition}
\newtheorem{theo}[thm]{Theorem}
\newtheorem{Cor}[thm]{Corollary}
\newtheorem{thm-defi}[thm]{Theorem-Definition}

\theoremstyle{definition}
\newtheorem{defi}[thm]{Definition}
\newtheorem{rem}[thm]{Remark}
\newtheorem{ex}[thm]{Example}
\newtheorem{Conc}[thm]{Conclusion}
\newtheorem{nota}[thm]{Notation}
\newtheorem{exer}[thm]{Exercise}
\newtheorem{soln}[thm]{Solution}
\usepackage{fancyhdr}

\begin{document}
\section{Revision}
\begin{prop}[Prop. 1.5 in Talk 8]
  Let $b_{ij} \in \Bbb Q_{p}$ and suppose $\forall i: \lim_{j \to \infty}b_{ij} = 0$ and $lim_{i \to \infty}b_{ij} = 0$ uniformly in $j$, then both series
  $\sum_{i=0}^{\infty}\left(\sum_{j=0}^{\infty} b_{ij} \right) \text{ and } \sum_{j=0}^{\infty}\left(\sum_{i=0}^{\infty} b_{ij} \right)$
  converge and have equal sum.
\end{prop}
\begin{prop}[Prop. 2.1 in Talk 8]
  Let $f(X) \in \Bbb Q_{p}[[X]]$ be a power series, then the radius of convergence is
  $$\rho = \frac{1}{\displaystyle \limsup_{n \to \infty}\sqrt[n]{|a_{n}|}}$$
\end{prop}
\section{Formal Derivatives of Power Series}
\begin{thm-defi}
  Let $f(X) = \sum_{n=0}^{\infty} a_{n}X^{n} $, we define its \textbf{formal derivative} as
  $$f'(X) = \sum_{n=1}^{\infty} na_{n}X^{n-1}, $$
  Then $f'(X)$ has the properties of the derivative:
  \begin{itemize}
    \item Additivity: $(f+g)'(X) = f'(X) + g'(X)$
    \item Product Rule: $(fg)'(X) = f(X)g'(X) + f'(X)g(x)$
    \item Chain Rule: $(f \circ g)'(X) = f'(g(X))g'(X)$
  \end{itemize}
\end{thm-defi}
\begin{proof} Let $f(X) =\sum_{n=0}^{\infty} a_{n}X^{n}, g(X) = \sum_{n=0}^{\infty} b_{n}X^{n}$
  \begin{itemize}
    \item Additivity:
          $$(f+g)'(X) = (\sum_{n=0}^{\infty} a_{n}X^{n} + \sum_{n=0}^{\infty} b_{n}X^{n})' = (\sum_{n=0}^{\infty} (a_{n}+b_{n})X^{n})' =$$
          $$\sum_{n = 1}^{\infty} n(a_{n} + b_{n})X^{n-1} = \sum_{n=1}^{\infty}na_{n}X^{n-1} + \sum_{n=1}^{\infty}nb_{n}X^{n-1} = f'(X) + g'(X)$$
    \item Product Rule:
          $$f(X)g'(X) + f'(X)g(X) =\left(\sum_{n = 0}^{\infty} a_{n}X^{n}\right) \cdot \left(\sum_{n=0}^{\infty}(n+1)b_{n+1}X^{n} \right) + \left(\sum_{n=0}^{\infty}(n+1)a_{n+1}X^{n}\right)\cdot \left(\sum_{n = 0}^{\infty} b_{n}X^{n}\right)$$
          $$ = \sum_{n =0}^{\infty}c_{n}X^{n} + \sum_{n =0}^{\infty}d_{n}X^{n},\quad c_{n} = \sum_{i=0}^{n}(i+1)b_{i+1}a_{n-i}, d_{n} = \sum_{i=0}^{n}(i+1)a_{i+1}b_{n-i}$$
          $$ = \sum_{n =0}^{\infty} \sum_{i=0}^{n}(i+1)b_{i+1}a_{n-i} X^{n}+ \sum_{n =0}^{\infty}\sum_{i=0}^{n}(i+1)a_{i+1}b_{n-i} X^{n} = \sum_{n=0}^{\infty}\left( \sum_{i=0}^{n}(i+1)b_{i+1}a_{n-i} + \sum_{i=0}^{n}(i+1)a_{i+1}b_{n-i}\right)X^{n}$$
          $$ = \sum_{n=0}^{\infty}\left( \sum_{i=0}^{n}(i+1)(b_{i+1}a_{n-i} + a_{i+1}b_{n-i})\right)X^{n}$$
    \item Chain Rule:
  \end{itemize}

  % TODO
\end{proof}
\begin{prop}
Let $f(X)$ be a power series which converges for all $|x| < \rho$, if $|a| < 1$ and $|b| < \rho$, then $g(x) = f(ax+b)$ is given by a power series $g(X)$ which converges for $|x| < \rho$.
\end{prop}
\section{Strassman's Theorem}
 % Let $f \in \Bbb Q_{p}[[X]]$ a formal power series over $\Bbb Q_{p}$, then for $x, y \in \Bbb Z_{p}$ we have
 % $$f(x) - f(y) = (x-y)\sum_{n=1}^{\infty}\sum_{j=0}^{n-1}a_{n}x^{j}y^{n-1-j}$$
 % $$f(x) - f(y) = \sum_{n=0}^{\infty}a_{n}x^{n} - \sum_{n=0}^{\infty}a_{n}y^{n} = \sum_{n=1}^{\infty}a_{n}(x^{n}-y^{n})$$
\begin{rem} % lemma 2.1
  Let $(R, +, \cdot)$ be a ring, $x, y \in R$, then we have $x^{n}-y^{n} = (x-y)\sum_{j=0}^{n-1}x^{j}y^{n-1-j}, \forall n \in \Bbb N_{0}$
  % TODO Ask if this should be in terms of rings or what
\end{rem}
\begin{proof} % proof 2.1
    , we use induction on $n$,
    Base case: $n = 2$, it's easy to see that
    $$(x-y)\sum_{j=0}^{n-1}x^{j}y^{n-1-j} = (x-y)(x+y) = x^{2} - y^{2}$$
    Induction hypothesis: we assume for an arbitrary $n \geq 2: x^{n}-y^{n} = (x-y)\sum_{j=0}^{n-1}x^{j}y^{n-1-j}$,
    Induction step: consider
    $$(x-y)\sum_{j=0}^{n}x^{j}y^{n-j} = (x-y)(y^{n} + y^{n-1}x + \cdots + x^{n-1}y + x^{n})$$
    $$ = (x-y)(y(y^{n-1} + y^{n-2}x + \cdots + yx^{n-2} + x^{n-1}) + x^{n}) = y\underbrace{(x-y)\sum_{j=0}^{n-1}x^{j}y^{n-1-j}}_{= x^{n}-y^{n}} + x^{n}(x-y)$$
    $$ = y(x^{n} - y^{n}) + x^{n}(x-y) = yx^{n} - y^{n+1} + x^{n+1} - yx^{n} = x^{n+1} - y^{n+1}.$$
\end{proof}
\begin{lemm}
  Let $f(X) \in \Bbb Q_{p}[[X]]$ be a non-zero power series which converges $\forall x \in \Bbb Z_{p}$, then $\exists N \in \Bbb N_{0}$ such that $|a_{N}| = \max_{n \in \Bbb N_{0}}|a_{n}| \text{ and } |a_{n}| < |a_{N}|\ \forall n > N$
\end{lemm}
\begin{proof}
  Since $f(X)$ converges $\forall x \in \Bbb Z_{p}$, then we have
  $$\forall x \in \Bbb Z_{p} : \lim_{n \to \infty}|a_{n}x^{n}| = 0 = \lim_{n \to \infty}|a_{n}|\cdot|x^{n}| \implies \lim_{n \to \infty}|a_{n}| = 0$$
\end{proof}
\begin{thm}[Strassman] % STRASSMAN
Let $f(X) \in \Bbb Q_{p}[[X]]$ and suppose we have $\lim_{n \to \infty}a_{n} = 0$, so that $f(x)$ converges $\forall x \in \Bbb Z_{p}$. Define $N \in \Bbb N_{0}$ like in Lemma 2.2
then the function $f$ has at most $N$ zeros.
\end{thm}
\begin{proof}
  induction on $N$.
  \begin{itemize}
    \item Base case: if $N = 0$, then $|a_{0}| > |a_{n}|, \forall n \geq 1$, we want to show that there are no zeros: $f(x) \neq 0 \forall x \in \Bbb Z_{p}$, if we had $f(x) = 0$, then
          $$0 = f(x) = a_{0} + a_{1}x + a_{2}x^{2} + \cdots$$
          $$ \implies |a_{0}| = |a_{1}x + a_{2}x^{2} + \cdots| \leq \max_{n \geq 1}|a_{n}x^{n}| \leq \max_{n \geq 1}|a_{n}|$$
          But this contradicts the assumption that $|a_{0}| > |a_{n}|, \forall n \geq 1$, so there are no zeros in this case.
    \item Induction step:
          Suppose $N$ was defined like before, and $\exists \alpha \in \Bbb Z_{p} : f(\alpha) = 0$, then we have for any $x \in \Bbb Z_{p}$
          $$f(x) = f(x) - f(\alpha) = \sum_{n=0}^{\infty}a_{n}x^{n} - \sum_{n=0}^{\infty}a_{n}\alpha^{n} = \sum_{n=0}^{\infty}a_{n}(x^{n} - \alpha^{n}) \overset{2.1}= (x-\alpha)\sum_{n=0}^{\infty}\sum_{j = 0}^{n-1} a_{n}x^{j}\alpha^{n-1-j}$$
          $$ = (x-\alpha) \sum_{n=0}^{\infty} \sum_{j=0}^{\infty} c_{nj}, \quad c_{nj}:= \begin{cases}a_{n}x^{j}\alpha^{n-1-j} & j < n, \\ 0 & j \geq n.\end{cases}$$
          We can use prop 1.1 to change the order of the summation but first we have to show the conditions of the proposition:
          \begin{enumerate}
            \item $\forall n \in \Bbb N_{0}, \lim_{j \to \infty} c_{nj} = 0$:
                  Clear, since we have $c_{nj} = 0, \forall j \geq n$.
            \item $\lim_{n \to \infty}c_{nj} = 0$ uniformly in $j$:
                  This is also easy to see, because we have $|a_{n}x^{j}\alpha^{n-1-j}| \leq |a_{n}| \to 0$ unrelated to $j$.
          \end{enumerate}
          So we can switch the sums and then we have
          $$(x-\alpha) \sum_{n=0}^{\infty} \sum_{j=0}^{\infty} c_{nj} = (x-\alpha) \sum_{j=0}^{\infty} \sum_{n=0}^{\infty} c_{nj}$$
          since $\forall j \geq n:c_{nj} = 0$, we need to only consider when $n > j$ so its equal to
          $$= (x-\alpha) \sum_{j=0}^{\infty} \sum_{n=j+1}^{\infty} a_{n}x^{j}\alpha^{n-1-j} = (x-\alpha) \sum_{j=0}^{\infty} x^{j} \underbrace{\sum_{n=0}^{\infty} a_{n+j+1}\alpha^{n}}_{=:b_{j}}$$
          $$= (x-\alpha)g(x),\quad g(x):=\sum_{j=0}^{\infty}b_{j}x^{j}$$
          Now we check if $g(X)$ fits the assumptions of the theorem, to use the induction steps. We need to show that $g(X)$ is non zero and that $b_{j} \to 0$
          \begin{itemize}
            \item $g(X)$ is non zero: clear since if $g(X)$ was the zero power series then $f(X)$ would also be zero, which is a contradiction.
            \item $b_{j} \to 0$: Consider $|b_{j}| = \left|\sum_{n=0}^{\infty}a_{n+j+1}\alpha^{n}\right| \leq \max_{n}|a_{n+j+1}\alpha^{n}| \leq \max_{n}|a_{n+j+1}| \xrightarrow{j \to \infty} 0$
          \end{itemize}
          Now we look for $\max_{j}|b_{j}|$, note that
          $$|b_{j}| \leq \max_{n}|a_{n+j+1}| \leq |a_{N}|, \forall j$$
          So we have
          $$|b_{N-1}| = |a_{N} + a_{N_+1}\alpha + a_{N+2}\alpha^{2} + \cdots | = |a_{N}|$$
          Finnaly, if $j > N-1$, then
          $$|b_{j}| \leq \max_{k}|a_{j+k+1}| \leq \max_{j > N}|a_{j}| < |a_{N}|$$
      So the index at which the maximum coefficient is reached $b_{n}$ is $N-1$, if we assume that $g(X)$ has at most $N-1$ zeros in $\Bbb Z_{p}$ then $f(X)$ has at most $N$ zeros (g's zeros and $\alpha$).
  \end{itemize}
\end{proof}
\begin{Cor} %5.6.2 in the book
  Let $f(X) = \sum a_{n}x^{n}$ be a non-zero power series which converges on $\Bbb Z_{p}$, and let $\alpha_{1}, ..., \alpha_{m} \in \Bbb Z_{p}$ be the roots of $f(X)$ in $\Bbb Z_{p}$, then there exists another power series $g(X)$ which also converges on $\Bbb Z_{p}$ but has no zeros in $\Bbb Z_{p}$, for which
  $$f(X) = \left(\prod_{i=1}^{m}(X-\alpha_{i})\right)g(X)$$
\end{Cor}
\begin{proof}
  Clear from the proof of the theorem.
\end{proof}

\begin{Cor}
  Let $f(X) = \sum a_{n}x^{n}$ be a non-zero power series which converges on $p^{m}\Bbb Z_{p}$, for some $m \in \Bbb Z$. Then $f(X)$ has a finite number of roots in $p^{m} \Bbb Z_{p}$.
\end{Cor}
\begin{proof}
  Define $$g(X) = f(p^{m}X) = \sum a_{n}p^{mn}X^{n},$$
  Since $f(x)$ converges for $x \in p^{m}\Bbb Z_{p}$, $g(x) = f(p^{m}x)$ converges for $x \in \Bbb Z_{p}$, applying the theorem to $g(X)$ gives the finiteness of its zeros.
\end{proof}
\begin{Cor}
  Let $f(X) = \sum a_{n}x^{n}$ and $g(X) = \sum b_{n}X^{n}$ be two p-adic power series which converge in a disc $p^{m}\Bbb Z_{p}$. If there exist infinitely many numbers $\alpha \in p^{m}Z_{p}$ such that $f(\alpha) = g(\alpha)$, then $a_{n} = b_{n}, \forall n \geq 0$
\end{Cor}
\begin{proof}
  Define $$h(X) = f(X) - g(X) = \sum(a_{n}-b_{n})X^{n}$$, then $h(X)$ converges also on $p^{m}\Bbb Z_{p}$, by Corollary 2.5 $h(X)$ has to have finitely many zeros, otherwise it must be the zero power series. Which means that
  $$f(X) = g(X) \implies a_{n} = b_{n} \forall n \geq 0$$
\end{proof}
\begin{Cor}
  Let $f(X) = \sum a_{n}x^{n}$ be a p-adic power series which converges in some disc $p^{m}\Bbb Z_{p}$. If the function $p^{m}\Bbb Z_{p} \to \Bbb Q_{p}, x \mapsto f(x)$ is periodic, that is, $\exists \pi \in p^{m}\Bbb Z_{p} : f(x + \pi) = f(x), \forall \in p^{m}\Bbb Z_{p}$ then $f(X)$ is constant.
\end{Cor}
\begin{proof}
The series $f(X) - f(0)$ has zeros at $n\pi$ for all $n \in \Bbb Z$, since $\pi \in p^{m}\Bbb Z_{p}$ implies $n\pi \in p^{m}\Bbb Z_{p}$, this gives infinitely many zeros, and hence the series $f(X) - f(0)$ must be identically zero, i.e. $f(X)$ must be constant.
\end{proof}
\begin{Cor}
  Let $f(X) = \sum a_{n}x^{n}$ be a p-adic power series which is entire, that is, $f(x)$ converges $\forall x \in \Bbb Q_{p}$. Then $f(X)$ has at most countably many zeros. Furthermore, if the set of zeros is not finite then the zeros form a sequence $\alpha_{n}$ with $|\alpha_{n}| \to \infty$.
\end{Cor}
\begin{proof}
  This is clear, because the number of zeros in each bounded disk $p^{m}\Bbb Z_{p}$ is finite.
\end{proof}
\section{The $p$-adic Logarithm Function}
\begin{defi}[Formal power series for the logarithm]   $$\mathbf{log}(1+X) = \sum_{n=1}^{\infty} (-1)^{n+1}\frac {X^{n}}{n} = X - \frac{X^{2}}{2} + \frac{X^{3}}{3} \mp \cdots \in \Bbb Q_{p}[[X]]$$
  Since the coefficients are in $\Bbb Q$ we can consider it as a power series with coefficients in $\Bbb Q_{p}$
\end{defi}
\begin{rem}
We use $\mathbf{log}$ when referring to the formal power series, not the logarithm function itself.
\end{rem}
\begin{prop} $\mathbf{log}(1+X)$ converges if and only if $|x| < 1$
\end{prop}
\begin{proof}
% TODO
\end{proof}
\begin{defi}Let $U_{1} = B(1,1) = \{x \in \Bbb Z_{p} : |x-1| < 1\} = 1+p\Bbb Z_{p}$, we define the $p$-adic logarithm of $x \in U_{1}$ as:
  $$\log_{p}(x) = \mathbf{log}(1+(x-1)) = \sum_{n=1}^{\infty} (-1)^{n+1}\frac {(x-1)^{n}}{n}$$
  In order to be able to call it a logarithm, it has to fill the usual logarithmic property:
\end{defi}
\begin{prop}
  Let $a, b \in 1+p\Bbb Z_{p}$, then we have
  $$\log_{p}(ab) = \log_{p}(a) + \log_{p}(b)$$
\end{prop}
\begin{proof}
  Let $x, y \in p\Bbb Z_{p}$ such that $a= 1+x, b =1+y$, and define for $x \in \Bbb pZ_{p}$
  $$f(x) = \log_{p}(1+x) = \sum_{n \geq 1}(-1)^{n+1}\frac {x^{n}}{n}$$
  % TODO
\end{proof}
\section{Roots of Unity}
\begin{prop}
  For $p \neq 2$ we have $\log_{p}(x)=0 \iff x=1$ and for $p = 2$, we have $\log_{p}(x)=0 \iff x= \pm 1$.
\end{prop}
\begin{proof}
  We know that $\log_{p}(x)$ converges only for $x \in p\Bbb Z_{p}$, not in $\Bbb Z_{p}$, but we can do a change of variables like in Corollary 2.5
  % TODO
\end{proof}
\begin{prop}
  Let $p \neq 2, x \in \Bbb Q_{p}$ and $x^{p}=1$, then $x=1$.
\end{prop}
\begin{proof}
% TODO
\end{proof}
\begin{Cor}
  (Remark 4.5 in Talk 6) There are no $p$-th and hence no $p^{n}$-th roots of unity in $\Bbb Q_{p}$.
\end{Cor}
\begin{proof}
% TODO
\end{proof}
\begin{prop} If $p = 2, x \in \Bbb Q_{2}$ and $x^{4} = 1$ then $x = \pm 1$, which means that there are no fourth roots of unity in $\Bbb Q_{2}$
\end{prop}
\begin{proof}
% TODO
\end{proof}
\begin{rem}
  We now summarize what we know so far about the roots of unity in $\Bbb Q_{p}$:
  \begin{itemize}
    \item If $p=2$, then the only roots of unity are $\pm 1$
    \item If $p \neq 2$, then $\Bbb Q_{p}$ contains all the $p-1$-st roots of unity and none other. (their existence was shown in Talk 6)
  \end{itemize}

\end{rem}

\begin{thebibliography}{widest-label} % References are listed here

	\bibitem[Gou]{Gou}%You can add more refernces by \bibitem command
	Fernando Q. Gouv\^{e}a:
	\emph{p-adic Numbers}.


\end{thebibliography}
\end{document}


