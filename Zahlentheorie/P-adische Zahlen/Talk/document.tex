\documentclass{article}
\usepackage[english]{babel}			
\usepackage[utf8]{inputenc}
\usepackage{geometry}
\usepackage[T1]{fontenc}
\usepackage{lmodern}
%\usepackage{tikz-cd}
\usepackage{graphicx}
\usepackage[colorlinks = true,
linkcolor = blue,
urlcolor  = blue,
citecolor = blue,
anchorcolor = blue]{hyperref}

\newcommand{\MYhref}[3][blue]{\href{#2}{\color{#1}{#3}}}%

\usepackage{amsmath}
\usepackage{amssymb}			
\usepackage{amsfonts}
\usepackage{amsthm}
\usepackage{mathtools}

\usepackage{tikz}
\usetikzlibrary{matrix}
\usetikzlibrary{fit}
\usetikzlibrary{backgrounds}
\usetikzlibrary{arrows}
\usetikzlibrary{shapes}


\theoremstyle{plain}
\newtheorem{thm}{Theorem}[section]
\newtheorem{lemm}[thm]{Lemma}
\newtheorem{prop}[thm]{Proposition} 
\newtheorem{theo}[thm]{Theorem}
\newtheorem{Cor}[thm]{Corollary}
\newtheorem{thm-defi}[thm]{Theorem-Definition}

\theoremstyle{definition}
\newtheorem{defi}[thm]{Definition}
\newtheorem{rem}[thm]{Remark}
\newtheorem{ex}[thm]{Example}
\newtheorem{Conc}[thm]{Conclusion}
\newtheorem{nota}[thm]{Notation}
\newtheorem{exer}[thm]{Exercise}
\newtheorem{soln}[thm]{Solution}
\usepackage{fancyhdr}

%%%%%%%%%%%%%%%%%%%%%%%%%%%%%%%%%%%%%%%%%%%%%%%%%%%%%%%
 
\begin{document}
\pagestyle{fancy}
\renewcommand{\footrulewidth}{0.4pt}
\fancyhead{}
\fancyhead[RO,RE]{$p-$adic numbers}
\fancyhead[LO,LE]{Yousef Khell}%This is the header
\fancyfoot{}
\fancyfoot[LO]{Handout} %This is the place for footnote
\begin{center}
	\Huge{\textbf{Talk 9: Strassman's theorem and the Logarithm function}}%Title spot
\end{center}
	\begin{minipage}[t]{0.49\textwidth}%This helps you write on one half of a page and may be helps you conserve some space
		\section{Revision}
		\begin{rem}\cite[Prop. 5.4.1]{Gou}
		Let $f(X) \in \Bbb Q_{p}[[X]]$ be a power series, then the radius of convergence is
		$\rho = \frac 1{(\limsup_{n \to \infty}\sqrt[n]{|a_{n}|})}$
		\end{rem}
		\begin{prop}{\cite[Prop. 5.1.4]{Gou}}
		Let $b_{ij} \in \Bbb Q_{p}$ and suppose $\forall i: \lim_{j \to \infty}b_{ij} = 0$ and $lim_{i \to \infty}b_{ij} = 0$ uniformly in $j$, then both series
		$\sum_{i=0}^{\infty}\left(\sum_{j=0}^{\infty} b_{ij} \right) \text{ and } \sum_{j=0}^{\infty}\left(\sum_{i=0}^{\infty} b_{ij} \right)$
		converge and have equal sum.
		\end{prop}
		\section{Strassman's Theorem}
		% Let $f \in \Bbb Q_{p}[[X]]$ a formal power series over $\Bbb Q_{p}$, then for $x, y \in \Bbb Z_{p}$ we have
		% $$f(x) - f(y) = (x-y)\sum_{n=1}^{\infty}\sum_{j=0}^{n-1}a_{n}x^{j}y^{n-1-j}$$
		% $$f(x) - f(y) = \sum_{n=0}^{\infty}a_{n}x^{n} - \sum_{n=0}^{\infty}a_{n}y^{n} = \sum_{n=1}^{\infty}a_{n}(x^{n}-y^{n})$$
		\begin{rem}
		Let $(R, +, \cdot)$ be a ring, $x, y \in R$, then we have $x^{n}-y^{n} = (x-y)\sum_{j=0}^{n-1}x^{j}y^{n-1-j}, \forall n \in \Bbb N_{0}$
		% TODO Ask if this should be in terms of rings or what
		\end{rem}
		\begin{lemm}
		Let $f(X) \in \Bbb Q_{p}[[X]]$ be a non-zero power series which converges $\forall x \in \Bbb Z_{p}$, then $\exists N \in \Bbb N_{0}$ such that $|a_{N}| = \max_{n \in \Bbb N_{0}}|a_{n}| \text{ and } |a_{n}| < |a_{N}|\ \forall n > N$
		\end{lemm}
		\begin{thm}[Strassman]
		Let $f(X) \in \Bbb Q_{p}[[X]]$ and suppose we have $\lim_{n \to \infty}a_{n} = 0$, so that $f(x)$ converges $\forall x \in \Bbb Z_{p}$. Define $N \in \Bbb N_{0}$ like in Lemma 2.2
		then the function $f$ has at most $N$ zeros.
		\end{thm}
		\begin{Cor} %5.6.2 in the book
		Let $f(X) = \sum a_{n}x^{n}$ be a non-zero power series which converges on $\Bbb Z_{p}$, and let $\alpha_{1}, ..., \alpha_{m} \in \Bbb Z_{p}$ be the roots of $f(X)$ in $\Bbb Z_{p}$, then there exists another power series $g(X)$ which also converges on $\Bbb Z_{p}$ but has no zeros in $\Bbb Z_{p}$, for which
		$$f(X) = \left(\prod_{i=1}^{m}(X-\alpha_{i})\right)g(X)$$
		\end{Cor}
	\end{minipage}%
	\hfill
	\begin{minipage}[t]{0.48\textwidth}
		\begin{Cor}
		Let $f(X) = \sum a_{n}x^{n}$ be a non-zero power series which converges on $p^{m}\Bbb Z_{p}$, for some $m \in \Bbb Z$. Then $f(X)$ has a finite number of roots in $p^{m} \Bbb Z_{p}$.
		\end{Cor}
		\begin{Cor}
		Let $f(X) = \sum a_{n}x^{n}$ and $g(X) = \sum b_{n}X^{n}$ be two p-adic power series which converge in a disc $p^{m}\Bbb Z_{p}$. If there exist infinitely many numbers $\alpha \in p^{m}Z_{p}$ such that $f(\alpha) = g(\alpha)$, then $a_{n} = b_{n}, \forall n \geq 0$
		\end{Cor}
		\begin{Cor}
		Let $f(X) = \sum a_{n}x^{n}$ be a p-adic power series which converges in some disc $p^{m}\Bbb Z_{p}$. If the function $p^{m}\Bbb Z_{p} \to \Bbb Q_{p}, x \mapsto f(x)$ is periodic, that is, $\exists \pi \in p^{m}\Bbb Z_{p} : f(x + \pi) = f(x), \forall \in p^{m}\Bbb Z_{p}$ then $f(X)$ is constant.
		\end{Cor}
		\begin{Cor}
		Let $f(X) = \sum a_{n}x^{n}$ be a p-adic power series which is entire, that is, $f(x)$ converges $\forall x \in \Bbb Q_{p}$. Then $f(X)$ has at most countably many zeros. Furthermore, if the set of zeros is not finite then the zeros form a sequence $\alpha_{n}$ with $|\alpha_{n}| \to \infty$.
		\end{Cor}
		\section{Formal Derivatives}
		\begin{thm-defi}
		Let $f(X) = \sum_{n=0}^{\infty} a_{n}X^{n} $, we define its \textbf{formal derivative} as
		$$f'(X) = \sum_{n=1}^{\infty} na_{n}X^{n-1}, $$
		Then $f'(X)$ has the properties of the derivative:
		\begin{itemize}
			\item Additivity: $(f+g)'(X) = f'(X) + g'(X)$
			\item Product rule: $(fg)'(X) = f(X)g'(X) + f'(X)g(x)$
			\item Chain rule: $(f \circ g)'(X) = f'(g(X))g'(X)$
		\end{itemize}
		\end{thm-defi}
	\end{minipage}
	\begin{minipage}[t]{0.45\textwidth}
		\begin{prop}
		Let $f(X)$ be a power series which converges for all $|x| < \rho$, if $|a| < 1$ and $|b| < \rho$, then $g(x) = f(ax+b)$ is given by a power series $g(X)$ which converges for $|x| < \rho$.
		\end{prop}
		\section{The $p$-adic Logarithm Function}
		\begin{defi}[Formal power series for the logarithm]   $$\mathbf{log}(1+X) = \sum_{n=1}^{\infty} (-1)^{n+1}\frac {X^{n}}{n} = X - \frac{X^{2}}{2} + \frac{X^{3}}{3} \mp \cdots$$
		Since the coefficients are in $\Bbb Q$ we can consider it as a power series with coefficients in $\Bbb Q_{p}$
		\end{defi}
		\begin{rem}
		We use $\mathbf{log}$ when referring to the formal power series, not the logarithm function itself.
		\end{rem}
		\begin{prop} $\mathbf{log}(1+X)$ converges if and only if $|x| < 1$
		\end{prop}
		\begin{defi}Let $U_{1} = B(1,1) = \{x \in \Bbb Z_{p} : |x-1| < 1\} = 1+p\Bbb Z_{p}$, we define the $p$-adic logarithm of $x \in U_{1}$ as:
		$$\log_{p}(x) = \mathbf{log}(1+(x-1)) = \sum_{n=1}^{\infty} (-1)^{n+1}\frac {(x-1)^{n}}{n}$$
		In order to be able to call it a logarithm, it has to fill the usual logarithmic property:
		\end{defi}
		\begin{prop}
		Let $a, b \in 1+p\Bbb Z_{p}$, then we have
		$$\log_{p}(ab) = \log_{p}(a) + \log_{p}(b)$$
		\end{prop}
		\section{Roots of Unity}
		\begin{prop}
		For $p \neq 2$ we have $\log_{p}(x)=0 \iff x=1$ and for $p = 2$, we have $\log_{p}(x)=0 \iff x= \pm 1$.
		\end{prop}
	\end{minipage}%
	\hfill
	\begin{minipage}[t]{0.45\textwidth}
		\begin{prop}
		Let $p \neq 2, x \in \Bbb Q_{p}$ and $x^{p}=1$, then $x=1$.
		\end{prop}
		\begin{Cor}
		(Remark 4.5 in Talk 6) There are no $p$-th and hence no $p^{n}$-th roots of unity in $\Bbb Q_{p}$.
		\end{Cor}
		\begin{prop} If $p = 2, x \in \Bbb Q_{2}$ and $x^{4} = 1$ then $x = \pm 1$, which means that there are no fourth roots of unity in $\Bbb Q_{2}$
		\end{prop}
		\begin{rem}
		We now summarize what we know so far about the roots of unity in $\Bbb Q_{p}$:
		\begin{itemize}
			\item If $p=2$, then the only roots of unity are $\pm 1$
			\item If $p \neq 2$, then $\Bbb Q_{p}$ contains all the $p-1$-st roots of unity and none other. (their existence was shown in Talk 6)
		\end{itemize}

		\end{rem}
		\begin{thebibliography}{widest-label} % References are listed here

			\bibitem[Gou]{Gou}%You can add more refernces by \bibitem command
			Fernando Q. Gouv\^{e}a:
			\emph{p-adic Numbers}.


		\end{thebibliography}
	\end{minipage}


\end{document}
